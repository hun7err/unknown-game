\chapter{Praca własna}
Proces twórczy został podzielony na dwa główne etapy: projektowanie i~implementację. Ponieważ tworzenie gry wymaga przygotowania oraz doboru odpowiedniego środowiska, etapy te zostały poprzedzone analizą problemu oraz burzą mózgów na której zarysowały się wstępne wymagania funkcjonalne. Zdefiniowane zostały również główne wymagania pozafunkcjonalne, takie jak docelowa platforma obsługująca grę. W celu zapewnienia spójnej wizji gry wśród członków zespołu, postanowiono skonstruować dokument, zawierający opis wszystkich decyzji podjętych podczas tworzenia projektu oraz wyjaśnienia dotyczące wszystkich wymagań funkcjonalnych oraz pozafunkcjonalnych -- Game Design Document (GDD), stanowiący jednocześnie załącznik 1 do niniejszej p~racy.  Pomysły zebrane podczas burzy mózgów zostały poddane wnikliwej analizie, co pozwoliło na stworzenie ogólna wersji projektu. 

\section{Projektowanie}

\subsection{Projekt silnika graficznego}

\subsection{Modele 3D}

\subsection{Projekt poziomu}

\section{Implementacja}

\subsection{Wykorzystanie technologii DirectX 11 w~UDK}

\subsection{Zarządzanie projektem}
W~celu sprawnej organizacji pracy w~zespole wykorzystano metody zarządzania projektami. 

Początkowo użyto modelu kaskadowego. Jego zaletą jest sekwencyjność, pozwalająca oddzielić procesy analizy problemu, projektowania, implementacji oraz testowania i~późniejszego utrzymania projektu. Metoda Waterfall, wykorzystująca ten model, nie jest jednak pozbawiona wad. 
Dotyczą one głównie dużych projektów, a~więc nie miały miejsca w~przypadku oprogramowania tworzonego w~czteroosobowym zespole programistów. Korzystając z~tej metody oszczędza się czas na planowaniu, faza ta zajmuje zaledwie 25\% czasu pracy nad projektem. Rezultat końcowy zostaje ustalony jeszcze przed rozpoczęciem implementacji, podobnie jak poszczególne przyrosty implementacji. Każdy przyrost miał trwać 2 tygodnie i~zawierał określone zadania. Rezultatem końcowym był produkt posiadający wartość biznesową, w~tym przypadku - w~pełni działająca gra. Co istotne, wartość biznesową produkt miał zyskać dopiero w~przedostatnim przyroście.

W~niedługim czasie po zaplanowaniu prac nad projektem okazało się, że metoda ta nie jest wystarczająca, zaczęło się pojawiać opóźnienie w~pracach, które mogło spowodować pogorszenie jakości produktu. Zdecydowano się więc skorzystać z~metod zwinnych. Metodyki Agile charakteryzują się stałą jakością, a~sterowane są zakresem. W~omówionej wcześniej metodzie Waterfall zakres był stały, zmieniała się jedynie jakość, a~chęć utrzymania wysokiej jakości powodowała opóźnienie względem planu. 

Ostatecznie zastosowano metodykę zwinną zbliżoną do metody Scrum. Zadania podzielono na część dotyczącą rozpoznania danego zagadnienia i~część implementacyjną. Realizowane były tygodniowe sprinty (przyrosty). Na koniec każdego z~nich otrzymywany był prototyp posiadający pewną funkcjonalność. Ze względu na wcześniejszy nieudany eksperyment z~metodą Waterfall wartość biznesową projekt zyskał dopiero po kliku przyrostach. Oznacza to, że pierwsze efekty nie były satysfakcjonujące, jednak prezentowały postęp prac. Każdy sprint rozpoczynał się spotkaniem zespołu, na którym omówiono pozostałe zadania oraz zaplanowano jakie funkcjonalności będą implementowane w~kolejnym przyroście. W~efekcie udało się zachować jakość wykonania, modyfikując nieznacznie zakres dostępnych funkcjonalności.

