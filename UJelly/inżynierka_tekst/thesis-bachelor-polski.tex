
% Szkielet dla pracy inżynierskiej pisanej w języku polskim.

\documentclass[polish,bachelor,a4paper,oneside]{ppfcmthesis}

\usepackage{polski}
\usepackage{pdfpages}
\usepackage[utf8]{inputenc}
\usepackage[OT4]{fontenc}
\usepackage[sectionbib]{chapterbib}

% Authors of the thesis here. Separate them with \and
\author{%
   Krzysztof~Marciniak \album{106574} \and 
   Piotr~Przybysz \album{106602} \and 
   Mikołaj~Szychowiak \album{106580} \and 
   Ryszard~Wojtkowiak \album{106609}}
\title{Projekt i implementacja gry komputerowej z~wykorzystaniem technologii Microsoft~DirectX~11}                   % Note how we protect the final title phrase from breaking
\ppsupervisor{dr~inż.Witold Andrzejewski} % Your supervisor comes here.
\ppyear{2015}                                         % Year of final submission (not graduation!)


\begin{document}
\sloppy
% Front matter starts here
\frontmatter\pagestyle{empty}%
\maketitle\cleardoublepage%

% Blank info page for "karta dyplomowa"
\thispagestyle{empty}\vspace*{\fill}%
\begin{center}Tutaj przychodzi karta pracy dyplomowej;\\oryginał wstawiamy do wersji dla archiwum PP, w pozostałych kopiach wstawiamy ksero.\end{center}%
\vfill\cleardoublepage%

% Table of contents.
\pagenumbering{Roman}\pagestyle{ppfcmthesis}%
\tableofcontents* \cleardoublepage%

\newcommand*{\captionsource}[2]{%
  \caption[{#1}]{%
    #1%
    \\\hspace{\linewidth}%
    \textsc{Źródło:} #2%
  }%
}

% Main content of your thesis starts here.
\mainmatter%

\chapter{Wst\k{e}p}

\section{Cel i~zakres pracy}

Celem pracy jest zaprojektowanie oraz zaimplementowanie gry komputerowej dla systemu Microsoft Windows, przy wykorzystaniu technologii Microsoft DirectX~11. W~ramach niniejszej pracy należy przygotować środowisko, głównie silnik graficzny, wykorzystujący wspomnianą wcześniej technologię. Przedmiotem pracy nie jest implementacja silnika graficznego, możliwe jest więc wykorzystanie istniejących. Narzędzia te mają różne charakterystyki i~nie każde dostępne na rynku będzie przydatne. Dlatego ważny jest wybór odpowiedniego narzędzia. 

Wszelkie informacje na temat  powstawania i~założeń projektu powinny być zapisane w~domumentacji. Stworzenie gry komputerowej wymaga również stworzenia modeli i~animacji. Aby umożliwić rozgrywkę konieczne jest również stworzenie skryptów opisujących mechanikę gry.  

Podczas prac ważna jest praca zespołowa. Tworzenie złożonego projektu od podstaw jest bardzo ważne dla inżyniera, stąd ważne jest wykorzystanie procesów projektowania i~wytwarzania oprogramowania w~ramach ninejszej pracy. Procesy te są czasochłonne. Istotne jest więc wykorzystanie metod zarządzania projektami w~celu minimalizacji ryzyka stworzenia wadliwego oprogramowania i~odpowiedniego zarządzania czasem, które pozwoli uniknąć zbędnych opóźnień. 

%Sprawdzić gdzie to ma byc, kto co robił i~czy tak może być ujęte
W~ramach niniejszej pracy Krzysztof Marciniak wykonał projekt silnika graficznego, modele 3D i~animacje. 
Piotr Przybysz zaprojektował poziom do gry i~wykonał interfejs użytkownika. 
Mikołaj Szychowiak był odpowiedzialny za zarządzanie projektem oraz rozpoznanie możliwości i~sposobów wykonania wymagań funkcjonalnych.
Ryszard Wojtkowiak zaprojektował broń i~stworzył skrypty mechaniki gry.
Wszyscy członkowie zespołu byli tak samo zaangażowani w~projektowanie oraz testowanie gry.

\chapter{Przegląd zagadnień teoretycznych}

\section{Zarys działania Microsoft DirectX API}

\subsection{Struktura i podstawowe pojęcia}

Microsoft DirectX to interfejs programistyczny (API, ang. \emph{application programming interface}) do tworzenia aplikacji multimedialnych. Składają się na nie przede wszystkim:

\begin{itemize}
\item DirectDraw, Direct2D/Direct3D (komponenty odpowiedzialne za rysowanie grafiki),
\item DirectSound, DirectMusic (obsługa dźwięków),
\item DirectInput (obsługa wejścia - myszy, klawiatury, kontrolerów itp.).
\end{itemize}

DirectX pozwala na tworzenie aplikacji w trzech językach - C\#, Visual Basic oraz C++, przy czym w praktyce spotyka się głównie aplikacje stworzone w C\# oraz C++.
Jak widać na rysunku \ref{directx_infrastructure}, struktura DirectX jest wielopoziomowa i, poza odwołaniami do API oraz komponentami jak na przykład Direct3D 11 (D3D11), obejmuje także DXGI (Infrastruktura Graficzna DirectX, ang. \emph{Microsoft DirectX Graphics Infrastructure}), które -- jako najniższa warstwa -- komunikuje się bezpośrednio ze sterownikiem znajdującym się w przestrzeni jądra systemu operacyjnego. Wykorzystanie komponentów COM (ang. \emph{Component Object Model}) pozwala na łatwą rozbudowę oraz jasny podział funkcjonalności.\\
\begin{figure}
\begin{center}
\includegraphics[scale=0.4]{figures/directx_infrastructure.png}
\caption{Architektura DirectX}
\label{directx_infrastructure}
\end{center}
\end{figure}
Podstawowymi i koniecznymi do zrozumienia działania DirectX są koncepcje urządzenia (ang. \emph{device}) oraz jego kontekstu (ang. \emph{device context}).\\
Urządzenie, reprezentowane w wersji 11. przez interfejs ID3D11Device, reprezentuje kartę graficzną i może służyć do tworzenia zasobów oraz pobierania informacji o jej możliwościach (ang. \emph{capabilities}), tj. oferowanych przez nią funkcjonalnościach takich jak np. obsługa podwójnej precyzji w programach cieniujących.\\
Kontekst urządzenia z kolei, jak wskazuje nazwa, określa kontekst użycia urządzenia, co jednoznacznie pokazuje, iż do prawidłowej pracy z urządzeniem powiązany powinien być co najmniej jeden kontekst. Pozwala on głównie na ustawienie stanu w procesie renderingu oraz ustawienie prawidłowych komend wykorzystujących zasoby karty graficznej do wygenerowania obrazu (bezpośrednio na ekran lub do pośredniczącej tekstury, która może zostać wykorzystana później). Wyróżniamy dwa rodzaje kontekstów: bezpośredni/zwykły (\emph{forward}) oraz opóźniony (\emph{deferred}). Bezpośredni wykonuje komendy od razu gdy zostają wywołane, podczas gdy opóźniony pozwala na zapisanie ich na odpowiedniej liście, dzięki czemu mogą zostać wykonane później (jest to przydatne zwłaszcza w przypadku aplikacji wielowątkowych).\\

\subsection{Proces renderingu}

Przebieg wygenerowania grafiki (renderingu) opisywany przez tzw. \emph{pipeline} jest złożony z wielu etapów (ang. \emph{stage}), z których część może być konfigurowana jedynie przez wywołanie odpowiednich komend API poprzedzonych prefiksem będącym skrótem nazwy danego etapu (np. \emph{IA} dla etapu \emph{Input-Assembler}), a część opisywana jest przez programy cieniujące (ang. \emph{shader}).\\
W pipeline DirectX 11 wyróżnia się następujące etapy:
\begin{itemize}
\item Input-Assembler - odpowiada za wczytanie wierzchołków w sposób opisany przez prymityw (trójkąt, czworokąt itp.),
\item Vertex Shader - cieniowanie wierzchołków, ustalanie dla nich wartości początkowych zmiennych jak np. kolor, wektor normalny itp. ,
\item Hull Shader - pierwszy etap teselacji (zagęszczania siatki), przygotowuje siatkę do zagęszczenia przez ustalenie punktów kontrolnych,
\item Tessellator - zagęszcza siatkę wprowadzając dodatkowe prymitywy zastępujące podstawowy i zwraca nowe współrzędne teksturowania,
\item Domain Shader - generuje nowe pozycje wierzchołków na podstawie dwóch poprzednich etapów,
\item Geometry Shader - pozwala zastąpić prymityw wejściowy innym prymitywem,
\item Rasterizer - odpowiada za "spłaszczenie" obrazu tak, aby mógł zostać wyświetlony na ekranie,
\item Pixel Shader - pozwala zmieniać kolor piksela w wynikowym obrazie,
\item Output Merger - generuje końcowy obraz łącząc w odpowiedni sposób (opisany przez komendy takie jak np. OMSetRenderTargets) informacje z poprzedniego etapu.
\end{itemize}

Wszystkie te etapy przedstawione zostały na Rys. \ref{directx_pipeline}. Strzałki określają czy dany etap korzysta jedynie z odczytu danych z karty, czy też pozwala na ich zapis (lub oba równocześnie).

\begin{figure}
\begin{center}
\includegraphics[scale=0.5]{figures/directx_pipeline.png}
\caption{Proces generowania obrazu w DirectX}
\label{directx_pipeline}
\end{center}
\end{figure}

Ostatnim zasługującym na uwagę jest fakt, iż w większości aplikacji wraz z DirectX wykorzystywana jest technologia WinAPI pozwalająca na tworzenie aplikacji graficznych pod platformę Windows. Głównym jej mechanizmem jest pętla komunikatów, których odebranie warunkuje sposób przetwarzania informacji w aplikacji (np. wciśnięcie klawisza powoduje przesunięcie modelu). Komunikaty mogą zostać odczytane w sposób blokujący (przez funkcję \emph{GetMessage}) lub nieblokujący (\emph{PeekMessage}). Z oczywistych względów (tj. narzutu czasowego), w aplikacjach generujących obraz w czasie rzeczywistym wykorzystywana jest wyłącznie funkcja nieblokująca.

\section{Deferred Shading}

Podstawowym sposobem obliczania oświetlenia, uwzględniając opisany wcześniej proces generacji obrazu, jest wyliczenie oświetlenia bezpośrednio dla każdego obiektu. Wartości takie jak wektory normalne, współrzędne teksturowania i inne używane we wspomnianym procesie są wczytywane w shaderze wierzchołków, a następnie interpolowane we fragment/pixel shaderze. W przypadku wielu źródeł oświetlenia (na przykład 200 lub więcej), dla każdego obiektu (niezależnie od tego, czy jest widoczny) należy sprawdzić wszystkie źródła światła, co oznacza złożoność
\begin{equation}
\label{forward_shading_complexity}
O(~liczba\_pikseli\_per\_obiekt~*~liczba\_zrodel\_swiatla~)
\end{equation}
co z kolei w wielu przypadkach jest nie do zaakceptowania (z \ref{forward_shading_complexity} wynika, iż zwiększanie liczby obiektów zmniejsza liczbę źródeł światła, które możemy użyć).\\
Jeśli jednak proces obliczania oświetlania przesuniemy do oddzielnego etapu następującego po wyliczeniu tego, które obiekty są aktualnie widoczne, otrzymamy złożoność
\begin{equation}
\label{deferred_shading_complexity}
O(~liczba\_pikseli~*~liczba\_zrodel\_swiatla~)
\end{equation}
Jak widać z \ref{deferred_shading_complexity}, podejście to nie wprowadza już zależności między liczbą obiektów a liczbą źródeł światła. Pozwala to uzyskać o wiele lepszą wydajność w przypadku scen z wieloma złożonymi obiektami oraz złożonym oświetleniem. Możliwe jest również wprowadzenie wielu uprawnień, jak na przykład podział obrazu na wiele "kafelek" (ang. \emph{tile}), z których każda może zostać obliczona przez wątki na karcie graficznej - technika ta nosi nazwę \emph{tiled deferred rendering} i jest powszechnie stosowana w popularnych silnikach graficznych.\\
Uzyskanie tego efektu wymaga utworzenia kilku tekstur pośredniczących, które łącznie mają nazwę \emph{G-Buffera}. Jest on wykorzystywany do przetrzymywania efektów pośrednich procesu renderingu oraz wyliczenia obrazu końcowego. Format G-Buffera zmienia się w zależności od zastosowania oraz algorytmu obliczania oświetlenia i osoby odpowiadającej za jego implementację, jednak najczęściej wykorzystywane są tekstury:
\begin{itemize}
\item wektorów normalnych,
\item koloru/albedo,
\item głębokości.
\end{itemize}
Pomimo wielu zalet, deferred shading ma jednak swoje wady. Obliczanie oświetlenia z gotowego obrazu uniemożliwia łatwe obliczenie kolorów w przypadku przezroczystych obiektów, zaś wyświetlanie obrazu końcowego w postaci tekstury wymaga wykorzystania bardziej skomplikowanych algorytmów obliczania antyaliasingu, co z kolei obciąża kartę graficzną. Mimo to deferred shading, w zmodyfikowanych odmianach, jest powszechnie wykorzystywany w wielu popularnych grach komputerowych oraz innych aplikacjach renderujących obraz w czasie rzeczywistym.

\section{Podpowierzchniowe Rozproszenie Światła (Subsurface Scattering)}

\section{Depth of Field}


\chapter{Przegląd narzędzi}
Jedną z~najważniejszych kwestii podczas tworzenia gry komputerowej jest prawidłowy dobór narzędzi, ponieważ decyduje to nie tylko o~komforcie pracy, ale także o~jakości końcowego produktu. Podczas tego procesu szczególny nacisk powinien zostać położony na doborze silnika graficznego, między innymi ze względu na dużą dywersyfikację narzędzi należących do tej kategorii oraz mnogość funkcjonalności przez nie oferowanych.

Ze względu na specyfikę tego rozdziału, w~kolejnych punktach opisane zostaną przetestowane rozwiązania wraz z~krótkim podsumowaniem w~formie listy jego zalet i wad. Warto w~tym miejscu również wspomnieć, iż wszystkie analizowane narzędzia oferują możliwość wykorzystania DirectX~11 API, choć niekoniecznie bezpośrednio.

\section{Własny silnik graficzny w~języku C++}
Proces doboru narzędzi rozpoczęto od zaprojektowania i~stworzenia własnego silnika graficznego w~języku C++ w~wersji 11 z wykorzystaniem środowiska Microsoft Visual Studio 2012 na platformę Windows. Pozwoliło to nie tylko na praktyczne wykorzystanie umiejętności nabytych podczas uczestnictwa w zajęciach z Inżynierii Oprogramowania, ale także poznać w praktyce wykorzystanie Microsoft DirectX API w~wersji 11 na najniższym dostępnym poziomie. Tworzenie tego rodzaju oprogramowania wymaga jednak nie tylko odpowiedniej ilości czasu, ale także dobrego zaprojektowania interakcji między klasami oraz zrozumienia wielu zagadnień z~zakresu grafiki komputerowej, w~większości takich, które wykraczają poza program przedmiotu Grafika Komputerowa i Wizualizacja.
Z~ostatnich dwóch powodów (oraz faktu, iż tworzenie silnika nie było tematem pracy inżynierskiej), najpierw ograniczono rozwój oprogramowania do jednej osoby, a~następnie zrezygnowano z~wykorzystania go w~pracy, uzasadniając tę decyzję wysoką czasochłonnością wytwarzania owego narzędzia.

Zalety:
\begin{itemize}
\item większe możliwości w~zakresie wykorzystania DirectX API
\item lepsza znajomość możliwości oferowanych przez oprogramowanie
\item brak kosztów
\end{itemize}

Wady
\begin{itemize}
\item czasochłonność
\item wysoki próg wejścia (znajomość m.in. C++ oraz DirectX API)
\item konieczność wytworzenia edytora poziomów i dodatnia odpowiednich funkcjonalności
\end{itemize}

\section{Unreal Engine 4}
Unreal Engine 4 (UE4) jest jednym z~najpopularniejszych silników graficznych dostępnych na rynku, co jest złożeniem wielu czynników. Pierwszym z~nich jest z~pewnością niska cena płatnej licencji (subskrypcja miesięczna to koszt x\$) oraz darmowy dostęp dla studentów zarówno w~ramach licencji edukacyjnej (należy w~tym wypadku zgłosić chęć wydania licencji w~ramach przedmiotu prowadzonego na uczelni) jak i~w~ramach GitHub Developers Pack (należy jedynie potwierdzić studencki adres e-mail oraz wykorzystać [ang. \emph{redeem}] licencję dostępną na odpowiedniej podstronie serwisu GitHub w~formie kodu [ang. \emph{serial code}]). Drugim jest jakość generowanych (renderowanych) obrazów -- wykorzystanie algorytmu Voxel Cone Tracing [na pewno ten?] (algorytm rozwiązywania zagadnienia globalnego oświetlenia w czasie rzeczywistym) pozwala uzyskać niemal fotorealistyczną grafikę, jednak kosztem wysokich wymagań sprzętowych.
Unreal Engine, zarówno w~wersji~3 jak i~4, oferuje dostęp do kodu źródłowego w~języku C++, co - po poznaniu API udostępnianego przez twórców - pozwala szybko i wygodnie rozwijać logikę gry. Umiejętność programowania nie jest jednak wymagana do tego ze względu na obecność mechanizmu blueprintów (znaleźć tłumaczenie), który pozwala tworzyć kod z wykorzystaniem bloków oferowanych bezpośrednio w~silniku graficznym (sprawdzić czy dokładnie tak jest).
Ze względu na wysokie wymagania sprzętowe (brak możliwości uruchomienia na komputerach laboratoryjnych oraz komputerach 75\% zespołu) ostatecznie odrzucono to rozwiązanie.

Zalety:
\begin{itemize}
\item tania (darmowa) licencja
\item wygodny edytor i~dostęp do API w~języku C++
\item niemal fotorealistyczna grafika
\end{itemize}


Wady:
\begin{itemize}
\item konieczność poznania API
\item wysokie wymagania sprzętowe
\end{itemize}

\section{Unity}

\section{OGRE}

\section{CryEngine 3}

\section{Unreal Engine 3}

\section{Unreal Development Kit (UDK)}

\section{Autodesk Maya}

\section{Autodesk 3D Studio Max}

\section{Blender}

\section{Adobe Flash - Scaleform}

\chapter{Praca własna}
Proces twórczy został podzielony na dwa główne etapy: projektowanie i~implementację. Ponieważ tworzenie gry wymaga przygotowania oraz doboru odpowiedniego środowiska, etapy te zostały poprzedzone analizą problemu oraz burzą mózgów na której zarysowały się wstępne wymagania funkcjonalne. Zdefiniowane zostały również główne wymagania pozafunkcjonalne, takie jak docelowa platforma obsługująca grę. W celu zapewnienia spójnej wizji gry wśród członków zespołu, postanowiono skonstruować dokument, zawierający opis wszystkich decyzji podjętych podczas tworzenia projektu oraz wyjaśnienia dotyczące wszystkich wymagań funkcjonalnych oraz pozafunkcjonalnych -- Game Design Document (GDD), stanowiący jednocześnie załącznik~1 do niniejszej pracy.  Pomysły zebrane podczas burzy mózgów zostały poddane wnikliwej analizie, co pozwoliło na stworzenie ogólna wersji projektu. 

\section{Projektowanie}
Projektowanie jest bardzo ważnym etapem prac nad każdym projektem informatycznym. Pozwala uspójnić wizję gry w~zespole oraz zdefiniować zadania, które będą wykonywane podczas implementacji. Dlatego dobrze, gdy na tym etapie pracy w proces twórczy zaangażowany jest każdy członek zespołu. 

Istotnym elementem tworzenia projektu gry jest wnikliwa analiza wymagań funkcjonalnych oraz uszeregowanie ich według wagi. Każdy z członków zespołu posiadał własną wizję gry, dlatego nie udało się osiągnąć spójnego uszeregowania wymagań. Połączenie wizji wszystkich twórców zaowocowało spójnym projektem, opisanym w~załączniku~1 -- GDD. 

Aby zapewnić, że projekt będzie spójny, konieczne jest stworzenie dokumentów projektowych. Powstrzymuje to programistów przed puszczaniem wodzy fantazji i~tworzeniem funkcjonalności niezgodnych z projektem. 

Na tym etapie przydatna okazała się nie tylko teoretyczna wiedza na temat tworzenia zaawansowanych projektów informatycznych, ale przede wszystkim doświadczenia innych twórców gier, które zespół poznał przy okazji udziału w~konferencjach takich jak Zjazd Twórców Gier (ZTG) czy World of Gamedev Knowledge (WGK). Zdobyto wiedzę potrzebną między innymi do stworzenia poziomów ciekawych dla graczy, wyboru funkcjonalności nie wymagających skomplikowanych i~często zawiłych implementacyjnie elementów (co często powoduje błędy i~trudności w~dalszym rozwoju projektu), jednocześnie będących skomplikowanymi z~punktu widzenia gracza. 

Dobrą praktyką przy tworzeniu gry jest również częste testowanie graficznego interfejsu użytkownika, jego czytelności i~łatwości użycia kluczowych funkcji w~ferworze walki. Pozwala to zaprojektować interfejs przyciągający wzrok oraz funkcjonalny. Projektując HUD [ang. \emph{Head-Up Display}], a~więc wszystkie istotne w~trakcie rozgrywki wskaźniki, mapkę, poziom życia; postanowiono rozmieścić interesujące dla gracza informacje analogicznie do popularnych gier z~gatunku strzelanek. Jest to atrakcyjne dla graczy, ponieważ nie muszą zmieniać swoich przyzwyczajeń by sprawdzić poziom życia. 

Jednak projektowanie to nie tylko uspójnienie rozgrywki, ale również specyfikacja dotycząca środowiska, w~którym gra powstanie. W tym celu należało wybrać odpowiednie narzędzia, co zostało opisane w~Rozdziale 3: Przegląd narzędzi. 

\subsection{Projekt silnika graficznego}

\subsection{Modele 3D}

\subsection{Projekt poziomu}

\section{Implementacja}

\subsection{Wykorzystanie technologii DirectX 11 w~UDK}

\subsection{Zarządzanie projektem}
W~celu sprawnej organizacji pracy w~zespole wykorzystano metody zarządzania projektami. Uspójnienie projektu gry podczas fazy projektowania pozwoliło na zdefiniowanie i~wyspecyfikowanie zadań, realizowanych podczas implementacji. Znając liczbę zadań, można było podzielić projekt na kolejne przyrosty. Projekt miał być realizowany z~wykorzystaniem metod zwinnych. Trudnością w~wykorzystaniu typowej zwinnej metody, takiej jak programowanie ekstremalne czy Scrum okazał się charakter projektu, będącego pracą dyplomową. Z~tego względu narzucony został ostateczny termin ukończenia produktu. Jest to sprzeczne z manifestem zwinności, dlatego zdecydowano się na inne rozwiązanie.

Początkowo użyto modelu kaskadowego. Jego zaletą jest sekwencyjność, pozwalająca oddzielić procesy analizy problemu, projektowania, implementacji oraz testowania i~późniejszego utrzymania projektu. Metoda Waterfall, wykorzystująca ten model, nie jest jednak pozbawiona wad. 
Dotyczą one głównie dużych projektów, a~więc nie miały miejsca w~przypadku oprogramowania tworzonego w~czteroosobowym zespole programistów. Korzystając z~tej metody oszczędza się czas na planowaniu, faza ta zajmuje zaledwie 25\% czasu pracy nad projektem. Rezultat końcowy zostaje ustalony jeszcze przed rozpoczęciem implementacji, podobnie jak poszczególne przyrosty implementacji. Każdy przyrost miał trwać 2 tygodnie i~zawierał określone zadania. Rezultatem końcowym był produkt posiadający wartość biznesową, w~tym przypadku - w~pełni działająca gra. Co istotne, wartość biznesową produkt miał zyskać dopiero w~przedostatnim przyroście.

W~niedługim czasie po zaplanowaniu prac nad projektem okazało się, że metoda ta nie jest wystarczająca, zaczęło się pojawiać opóźnienie w~pracach, które mogło spowodować pogorszenie jakości produktu. Zdecydowano się więc skorzystać z~metod zwinnych. Metodyki Agile charakteryzują się stałą jakością, a~sterowane są zakresem. W~omówionej wcześniej metodzie Waterfall zakres był stały, zmieniała się jedynie jakość, a~chęć utrzymania wysokiej jakości powodowała opóźnienie względem planu. 

Ostatecznie zastosowano metodykę zwinną zbliżoną do metody Scrum. Zadania podzielono na część dotyczącą rozpoznania danego zagadnienia i~część implementacyjną. Realizowane były tygodniowe sprinty (przyrosty). Na koniec każdego z~nich otrzymywany był prototyp posiadający pewną funkcjonalność. Ze względu na wcześniejszy nieudany eksperyment z~metodą Waterfall wartość biznesową projekt zyskał dopiero po kliku przyrostach. Oznacza to, że pierwsze efekty nie były satysfakcjonujące, jednak prezentowały postęp prac. Każdy sprint rozpoczynał się spotkaniem zespołu, na którym omówiono pozostałe zadania oraz zaplanowano jakie funkcjonalności będą implementowane w~kolejnym przyroście. W~efekcie udało się zachować jakość wykonania oraz zakończyć prace przed upływem niezmiennego terminu ukończenia, modyfikując nieznacznie zakres dostępnych funkcjonalności.
\chapter{Podsumowanie}

Celem pracy inżynierskiej było stworzenie gry komputerowej. Proces ten okazał się długotrwały i~skomplikowany. Istotne zatem było podzielenie pracy na kilka części, czego dokonano równomiernie aby móc sprawiedliwie rozdzielić prace. Dzięki temu żaden z~członków grupy nie nadpisywał pracy innego i~możliwe było działanie równoległe. Tworzenie gry wymagało korzystania z~technologii i~programów nie będących w programie studiów, jednak dzięki temu poszerzono wiedzę członków zespołu. Zaobserwowano również efekty pracy nad grą w~profesjonalnym środowisku, którego używają na co dzień firmy produkujące tytuły na masową skalę. Aby zastosować podobne podejście, utworzono dokument projektowy określający projekt i~implementację gry pod nazwą ''Game Design Document'', który stanowi załącznik do pracy oraz wykorzystano niektóre metodyki programowania zwinnego. Dzięki temu znacznie przyspieszono pracę, choć utrapieniem były problemy z konfiguracją i~środowiskiem. Wybór odpowiedniego nie był łatwym zadaniem, co spowodowane było koniecznością przetestowania wielu z~nich w~praktyce. Niektóre trzeba było odrzucić z~powodu zbyt niskich możliwości sprzętowych. To tylko dowodzi, iż~trudno tworzyć gry w~nowej technologii bez drogiego sprzętu.

Stworzenie własnej gry komputerowej dostarcza wielu cennych porad. Dzięki uczeniu się na własnych błędach i~odpowiednim podejściu do pracy można znacznie przyspieszyć ten proces. Przykładem może być fakt, iż nie da się znaleźć odpowiedniego środowiska bez licznych testów, dzięki którym wiadome było co jest istotne i~czego potrzeba w grze, a~jakie są granice środowiska graficznego. W związku z tym istotny był wybór odpowiedniego -- takiego, które daje najwięcej możliwości za jak najniższą cenę.
% Bibliography (books, articles) starts here.
\chapter{Literatura}
\renewcommand{\section}[2]{}
\bibliographystyle{plalpha}{\raggedright\sloppy\small\bibliography{bibliography}{}}
%\bibliographystyle{plalpha}{\raggedright\sloppy\small\bibliography{bibliography}}
\chapter{Dodatki}

\appendix
\chapter{Game Design Document}
\label{chap:gdd_appendix}

\includepdf[pages={1,3-12}]{game_design_document.pdf}

% All appendices and extra material, if you have any.
\cleardoublepage\appendix%


% Colophon is a place where you should let others know about copyrights etc.
\ppcolophon

\end{document}
