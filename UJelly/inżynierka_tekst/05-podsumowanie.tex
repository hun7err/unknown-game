\chapter{Podsumowanie}

Celem pracy inżynierskiej było stworzenie gry komputerowej. Proces ten okazał się długotrwały i wyboisty. Istotne zatem było podzielenie pracy na kilka części, które równomiernie podzielono. Dzięki temu żaden z członków grupy nie nadpisywał pracy innego i każdy mógł działać równolegle. Tworzenie gry wymagało korzystania z technologii i programów nie przerabianych na studiach. Dzięki temu poszerzono swoje horyzonty. Zobaczono jak to jest pracować nad grą w profesjonalnym środowisku którego używają na co dzień firmy produkujące produkcje na masową skalę. Postanowiono stworzyć grę tak jak to robią wielkie firmy, więc stworzono GDD. Wykorzystano również niektóre metodyki zwinnego programowania. Dzięki temu znacznie przyspieszono pracę. Utrapieniem były problemy z konfiguracją i środowiskiem. Wybór odpowiedniego nie był łatwym zadaniem, spowodowało to konieczność przetestowania wielu z nich. Niektóre trzeba było odrzucić z powodu zbyt niskich możliwości sprzętowych. To tylko dowodzi, że trudno tworzyć gry w nowej technologii bez drogiego sprzętu.

Stworzenie własnej gry komputerowej dostarcza wielu cennych porad. Dzięki uczeniu się na własnych błędach i odpowiednim podejściu do pracy, można przyspieszyć ten proces. Na przykład, nie da się znaleźć odpowiedniego środowiska bez licznych testów, dzięki którym wiadome było co jest istotne i czego potrzeba w grze a jakie są granice środowiska graficznego które nas ograniczają, w związku z tym istotny był wybór odpowiedniego, takiego które daje najwięcej możliwości za jak najniższą cenę.