\chapter{Podsumowanie}

Celem pracy inżynierskiej było stworzenie gry komputerowej. Proces ten okazał się długotrwały i~skomplikowany. Istotne zatem było podzielenie pracy na kilka części, czego dokonano równomiernie aby móc sprawiedliwie rozdzielić prace. Dzięki temu żaden z~członków grupy nie nadpisywał pracy innego i~możliwe było działanie równoległe. Tworzenie gry wymagało korzystania z~technologii i~programów nie będących w~programie studiów, jednak dzięki temu poszerzono wiedzę członków zespołu. Zaobserwowano również efekty pracy nad grą w~profesjonalnym środowisku, którego używają na co dzień firmy produkujące tytuły na masową skalę. Aby zastosować podobne podejście, utworzono dokument projektowy określający projekt i~implementację gry pod nazwą ''Game Design Document'', który stanowi załącznik do pracy oraz wykorzystano niektóre metodyki programowania zwinnego. Dzięki temu znacznie przyspieszono pracę, choć trudnościami były problemy z~konfiguracją i~środowiskiem. Wybór odpowiedniego nie był łatwym zadaniem, co spowodowane było koniecznością przetestowania wielu z~nich w~praktyce. Niektóre trzeba było odrzucić z~powodu zbyt niskich możliwości sprzętowych. To tylko dowodzi, iż~trudno tworzyć gry w~nowej technologii bez drogiego sprzętu.

Stworzenie własnej gry komputerowej dostarcza wielu cennych porad. Dzięki uczeniu się na własnych błędach i~odpowiednim podejściu do pracy można znacznie przyspieszyć ten proces. Przykładem może być fakt, iż nie da się znaleźć odpowiedniego środowiska bez licznych testów, dzięki którym wiadome było co jest istotne i~czego potrzeba w~grze, a~jakie są granice środowiska graficznego. W~związku z~tym istotny był wybór odpowiedniego -- takiego, które daje najwięcej możliwości za jak najniższą cenę.

W~przyszłości planowany jest rozwój gry. Przydatny byłby tryb gry jednoosobowej. Wymagałoby to stworzenia wielu ciekawych poziomów oraz zaimplementowanie nietrywialnej sztucznej inteligencji.  Dla wielu graczy ciekawy może okazać się tryb fabularny, opowiadający historię młodego żelka, próbującego uniknąć zjedzenia przez ludzi.

W~grze wieloosobowej planowane jest dodanie dodatkowych trybów rozgrywki, takich jak przechwyć flagę [ang. \emph{Capture the Flag}] czy inne, wymagające współpracy wielu graczy. Pozwoli to urozmaicić rozgrywkę, ale również pomoże młodym graczom odkryć uroki pracy zespołowej.

Opisane w Game Design Document plany promocji gry, zakładają zgłoszenie gry do konkursu Imagine Cup, organizowanego przez Microsoft. Pomocna w~dalszym rozwoju gry oraz umiejętności twórców może być prezentacja gry na konferencjach poświęconych tworzeniu gier komputerowych. Pozwoli to uzyskać opinie osób, które tworzą elektroniczną rozrywkę zawodowo. Ich doświadczenie i rady z~pewnością przyczyni się do dalszego rozwoju twórców gry.