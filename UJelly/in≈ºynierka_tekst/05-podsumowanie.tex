\chapter{Podsumowanie}

Celem pracy inżynierskiej było stworzenie gry komputerowej. Proces ten okazał się długotrwały i~skomplikowany. Istotne zatem było podzielenie pracy na kilka części, czego dokonano równomiernie aby móc sprawiedliwie rozdzielić prace. Dzięki temu żaden z~członków grupy nie nadpisywał pracy innego i~możliwe było działanie równoległe. Tworzenie gry wymagało korzystania z~technologii i~programów nie będących w~programie studiów, jednak dzięki temu poszerzono wiedzę członków zespołu. Zaobserwowano również efekty pracy nad grą w~profesjonalnym środowisku, którego używają na co dzień firmy produkujące tytuły na masową skalę. Aby zastosować podobne podejście, utworzono dokument projektowy określający projekt i~implementację gry pod nazwą ''Game Design Document'', który stanowi załącznik do pracy oraz wykorzystano niektóre metodyki programowania zwinnego. Dzięki temu znacznie przyspieszono pracę, choć utrapieniem były problemy z~konfiguracją i~środowiskiem. Wybór odpowiedniego nie był łatwym zadaniem, co spowodowane było koniecznością przetestowania wielu z~nich w~praktyce. Niektóre trzeba było odrzucić z~powodu zbyt niskich możliwości sprzętowych. To tylko dowodzi, iż~trudno tworzyć gry w~nowej technologii bez drogiego sprzętu.

Stworzenie własnej gry komputerowej dostarcza wielu cennych porad. Dzięki uczeniu się na własnych błędach i~odpowiednim podejściu do pracy można znacznie przyspieszyć ten proces. Przykładem może być fakt, iż nie da się znaleźć odpowiedniego środowiska bez licznych testów, dzięki którym wiadome było co jest istotne i~czego potrzeba w~grze, a~jakie są granice środowiska graficznego. W~związku z~tym istotny był wybór odpowiedniego -- takiego, które daje najwięcej możliwości za jak najniższą cenę.