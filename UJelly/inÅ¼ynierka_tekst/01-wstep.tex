
\chapter{Wst\k{e}p}

\section{Cel i~zakres pracy}

Celem pracy jest zaprojektowanie oraz zaimplementowanie gry komputerowej dla systemu Microsoft Windows, przy wykorzystaniu technologii Microsoft DirectX~11. W~ramach niniejszej pracy należy przygotować środowisko, głównie silnik graficzny, wykorzystujący wspomnianą wcześniej technologię. Przedmiotem pracy nie jest implementacja silnika graficznego, możliwe jest więc wykorzystanie istniejących. Narzędzia te mają różne charakterystyki i~nie każde dostępne na rynku będzie przydatne. Dlatego ważny jest wybór odpowiedniego narzędzia. 

Wszelkie informacje na temat  powstawania i~założeń projektu powinny być zapisane w~domumentacji. Stworzenie gry komputerowej wymaga również stworzenia modeli i~animacji. Aby umożliwić rozgrywkę konieczne jest również stworzenie skryptów opisujących mechanikę gry.  

Podczas prac ważna jest praca zespołowa. Tworzenie złożonego projektu od podstaw jest bardzo ważne dla inżyniera, stąd ważne jest wykorzystanie procesów projektowania i~wytwarzania oprogramowania w~ramach ninejszej pracy. Procesy te są czasochłonne. Istotne jest więc wykorzystanie metod zarządzania projektami w~celu minimalizacji ryzyka stworzenia wadliwego oprogramowania i~odpowiedniego zarządzania czasem, które pozwoli uniknąć zbędnych opóźnień. 

%Sprawdzić gdzie to ma byc, kto co robił i~czy tak może być ujęte
W~ramach niniejszej pracy Krzysztof Marciniak wykonał projekt silnika graficznego, modele 3D i~animacje. 
Piotr Przybysz zaprojektował poziom do gry i~wykonał interfejs użytkownika. 
Mikołaj Szychowiak był odpowiedzialny za zarządzanie projektem oraz rozpoznanie możliwości i~sposobów wykonania wymagań funkcjonalnych.
Ryszard Wojtkowiak zaprojektował broń i~stworzył skrypty mechaniki gry.
Wszyscy członkowie zespołu byli tak samo zaangażowani w~projektowanie oraz testowanie gry.