
\chapter{Wst\k{e}p}

Przemysł gier komputerowych jest dynamicznym sektorem gospodarki. Stosunkowo niedawno gry były ciekawostką, zaś obecnie są produktem, w~który inwestowanych są środki sięgające milionów dolarów, nad którym pracują setki osób i przy których powstają całe społeczności graczy z~całego świata. Gry zyskały realny wpływ na kulturę -- zajmują w niej coraz ważniejsze miejsce, czego przykładem jest powstanie międzynarodowych turniejów sportu elektronicznego. Rynek gier stymuluje także rozwój technologii dotyczących komputerów osobistych, stawiając im coraz większe wymagania przez najnowsze tytuły, które posiadają coraz lepszą oprawę graficzną czy dźwiękową.

Samo tworzenie gier jest ciekawym zjawiskiem, ponieważ zawiera w~sobie potrzebę zarządzania zespołem i~jego zasobami, pracę jako programista, grafik, tester czy też projektant poziomów. Niektórzy, zachęceni sukcesami dużych gier, próbowali własnych sił w~mniejszych zespołach lub nawet pojedynczo oraz przy ograniczonym budżecie, aby stworzyć grę.
Gry powstałe w~ten sposób nazywa się grami niezależnymi, które obecnie odnoszą coraz większe sukcesy i~zdobywają popularność.

Popularnością cieszą się również różnego rodzaju konkursy. Niektóre z~nich polegają na stworzeniu prostej gry w~24 lub 48 godzin (tzw. \emph{game jam}), lecz są również bardziej tradycyjne, polegające na ocenianiu gotowego produktu tworzonego przez dłuższy czas. Sam udział w~takim konkursie pozwala twórcy zaistnieć, a~jego grze -- zdobyć popularność, co może okazać się początkiem kariery projektanta gier komputerowych.



\section{Cel i~zakres pracy}

Celem pracy jest zaprojektowanie oraz zaimplementowanie gry komputerowej dla systemu Microsoft Windows, przy wykorzystaniu technologii Microsoft DirectX~11. W~ramach niniejszej pracy należy przygotować środowisko, głównie silnik graficzny, wykorzystujący wspomnianą wcześniej technologię. Narzędzia te mają różne charakterystyki i~nie każde dostępne na rynku będzie przydatne, dlatego ważny jest wybór odpowiedniego. 

Wszelkie informacje na temat  powstawania i~założeń projektu powinny być zapisane w~domumentacji. Stworzenie gry komputerowej wymaga również stworzenia modeli i~animacji. Aby umożliwić rozgrywkę konieczne jest również stworzenie skryptów opisujących mechanikę gry.  

Podczas prac ważna jest praca zespołowa. Tworzenie złożonego projektu od podstaw jest bardzo ważne dla inżyniera, stąd ważne jest wykorzystanie procesów projektowania i~wytwarzania oprogramowania w~ramach ninejszej pracy. Procesy te są czasochłonne. Istotne jest więc wykorzystanie metod zarządzania projektami w~celu minimalizacji ryzyka stworzenia wadliwego oprogramowania i~odpowiedniego zarządzania czasem, które pozwoli uniknąć zbędnych opóźnień. 

%Sprawdzić gdzie to ma byc, kto co robił i~czy tak może być ujęte
W~ramach niniejszej pracy Krzysztof Marciniak wykonał projekt silnika graficznego, modele 3D i~animacje. 
Piotr Przybysz zaprojektował poziom do gry i~wykonał interfejs użytkownika. 
Mikołaj Szychowiak był odpowiedzialny za zarządzanie projektem oraz rozpoznanie możliwości i~sposobów wykonania wymagań funkcjonalnych.
Ryszard Wojtkowiak zaprojektował broń i~stworzył skrypty mechaniki gry.
Wszyscy członkowie zespołu byli tak samo zaangażowani w~projektowanie oraz testowanie gry.

\section{Struktura pracy}
Niniejsza praca stanowi opis procesu powstawania gry komputerowej wykorzystującej nowoczesne technologie. Przedstawia etapy tworzenia złożonego projektu począwszy od pomysłu, poprzez analizę funkcjonalności, projektowanie po implementację. 

Struktura pracy jest następująca:
\begin{itemize}
\item Rozdział drugi opisuje zagadnienia teoretyczne, będące podstawą do stworzenia gry komputerowej. Zebrane zostały tutaj informacje na temat wykorzystania technologii Microsoft DirectX~11 API oraz zagadnienia z~zakresu grafiki komputerowej.
\item W~rozdziale trzecim zawarty został opis narzędzi wykorzystanych podczas tworzenia gry. Opisane są tutaj również narzędzia, które nie zostały wykorzystane, a~które były rozważane jako przydatne. 
\item Czwarty rozdział stanowi opis pracy włożonej w~proces tworzenia gry. Zawarty jest tutaj opis projektowania gry oraz proces implementacji. Znajduje się tu również opis zarządzania projektem.
\item Rozdział piąty zawiera wnioski, uwagi oraz opis planowanego rozwoju i~promocji gry.
\item Szósty rozdział to bibliografia wykorzystana podczas tworzenia pracy.
\item Zawartym w~rozdziale siódmym dodatkiem jest, powstający równolegle z~grą, dokument projektowy -- Game Design Document (GDD) zawierający opis projektu oraz decyzje podjęte podczas tworzenia gry.
\end{itemize}