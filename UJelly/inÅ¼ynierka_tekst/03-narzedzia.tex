
\chapter{Przegląd narzędzi}
Jedną z~najważniejszych kwestii podczas tworzenia gry komputerowej jest prawidłowy dobór narzędzi, ponieważ decyduje to nie tylko o~komforcie pracy, ale także o~jakości końcowego produktu. Podczas tego procesu szczególny nacisk powinien zostać położony na doborze silnika graficznego, między innymi ze względu na dużą dywersyfikację narzędzi należących do tej kategorii oraz mnogość funkcjonalności przez nie oferowanych.

Ze względu na specyfikę tego rozdziału, w~kolejnych punktach opisane zostaną przetestowane rozwiązania wraz z~krótkim podsumowaniem w~formie listy jego zalet i wad. Warto w~tym miejscu również wspomnieć, iż wszystkie analizowane narzędzia oferują możliwość wykorzystania DirectX~11 API, choć niekoniecznie bezpośrednio.

\section{Własny silnik graficzny w~języku C++}
Proces doboru narzędzi rozpoczęto od zaprojektowania i~stworzenia własnego silnika graficznego w~języku C++ w~wersji 11 z wykorzystaniem środowiska Microsoft Visual Studio 2012 na platformę Windows. Pozwoliło to nie tylko na praktyczne wykorzystanie umiejętności nabytych podczas uczestnictwa w zajęciach z Inżynierii Oprogramowania, ale także poznać w praktyce wykorzystanie Microsoft DirectX API w~wersji 11 na najniższym dostępnym poziomie. Tworzenie tego rodzaju oprogramowania wymaga jednak nie tylko odpowiedniej ilości czasu, ale także dobrego zaprojektowania interakcji między klasami oraz zrozumienia wielu zagadnień z~zakresu grafiki komputerowej, w~większości takich, które wykraczają poza program przedmiotu Grafika Komputerowa i Wizualizacja.
Pomimo iż stworzenie własnego silnika oferuje największą swobodę w tworzeniu gry, to podejście okazało się zbyt pracochłonne, aby można je było wykorzystać do realizacji zadanego tematu pracy inżynierskiej w~dostępnym czasie. Na czasochłonność miała wpływ m.in. złożoność tworzonego oprogramowania oraz konieczność wdrażania się członków zespołu w~stworzony przez siebie kod.\\

Zalety:
\begin{itemize}
\item większe możliwości w~zakresie wykorzystania DirectX API,
\item lepsza znajomość możliwości oferowanych przez oprogramowanie,
\item brak kosztów.
\end{itemize}

Wady:
\begin{itemize}
\item czasochłonność,
\item wysoki próg wejścia (znajomość m.in. C++ oraz DirectX API),
\item konieczność wytworzenia edytora poziomów i dodatnia odpowiednich funkcjonalności.
\end{itemize}

\section{Unreal Engine 4}
Unreal Engine 4 (UE4) jest jednym z~najpopularniejszych silników graficznych dostępnych na rynku, co jest złożeniem wielu czynników. Pierwszym z~nich jest z~pewnością niska cena płatnej licencji (subskrypcja miesięczna to koszt 19\$) oraz darmowy dostęp dla studentów zarówno w~ramach licencji edukacyjnej (należy w~tym wypadku zgłosić chęć wydania licencji w~ramach przedmiotu prowadzonego na uczelni) jak i~w~ramach GitHub Developers Pack (należy jedynie potwierdzić studencki adres e-mail oraz wykorzystać [ang. \emph{redeem}] licencję dostępną na odpowiedniej podstronie serwisu GitHub w~formie kodu [ang. \emph{serial code}]). Drugim jest jakość generowanych (renderowanych) obrazów -- wykorzystanie algorytmu Light Propagation Volumes (algorytm rozwiązywania zagadnienia globalnego oświetlenia w czasie rzeczywistym) pozwala uzyskać niemal fotorealistyczną grafikę, jednak kosztem wysokich wymagań sprzętowych.
Unreal Engine, zarówno w~wersji~3 jak i~4, oferuje dostęp do kodu źródłowego w~języku C++, co - po poznaniu API udostępnianego przez twórców - pozwala szybko i wygodnie rozwijać logikę gry. Umiejętność programowania nie jest jednak wymagana do tego ze względu na obecność mechanizmu blueprintów, który pozwala tworzyć kod z wykorzystaniem bloków oferowanych bezpośrednio w~silniku graficznym.
Ze względu na wysokie wymagania sprzętowe (brak możliwości uruchomienia na komputerach laboratoryjnych oraz komputerach 75\% zespołu) ostatecznie odrzucono to rozwiązanie.\\

Zalety:
\begin{itemize}
\item tania (darmowa) licencja,
\item wygodny edytor i~dostęp do API w~języku C++,
\item niemal fotorealistyczna grafika.
\end{itemize}

Wady:
\begin{itemize}
\item konieczność poznania API,
\item wysokie wymagania sprzętowe.
\end{itemize}

\section{Unity}

Unity Engine to obecnie najpopularniejszy silnik graficzny wśród twórców gier niezależnych. Poza płatną licencją oferuje także licencję darmową z okrojoną listą funkcjonalności, które jednak nadal pozwalają na tworzenie dość zaawansowanych gier (wycięte funkcjonalności związane są głównie z jakością grafiki). Ma on także niski próg wejścia ze względu na zastosowanie w skryptach języka C\# lub JavaScript oraz oferuje 30-dniową wersję próbną. Odrzucono go głównie ze względu na wysoką cenę wersji Pro oraz brak pewnych istotnych funkcjonalności (brak miękkich cieni, brak cieni dla źródeł światła innych niż punktowe itp.)\\

Zalety:
\begin{itemize}
\item darmowa (choć okrojona) wersja,
\item niski próg wejścia,
\item 30-dniowa wersja próbna.
\end{itemize}

Wady:
\begin{itemize}
\item gorsza wydajność w stosunku do innych silników (języki C\# i JavaScript oraz wykorzystanie dynamicznych komponentów w trakcie działania powodują powstanie dodatkowego narzutu czasowego),
\item wysoka cena licencji Pro (1500\$),
\item brak pewnych istotnych dla jakości funkcjonalności.
\end{itemize}

\section{OGRE}

Object-Oriented Graphics Rendering Engine to elastyczny silnik graficzny o~otwartym kodzie źródłowym napisany w~języku C++. Pozwala na korzystanie z~dwóch API: OpenGL oraz Microsoft DirectX, w tym DirectX 11, przez co został wybrany jako alternatywa dla DirectX 11. Niestety nie jest on rozwijany na bieżąco, przez co wsparcie dla DirectX 11 jest jedynie częściowe i~w~większości wypadków iluzoryczne, ponieważ próba stworzenia urządzenia kończy się rzuceniem wyjątku wewnątrz biblioteki. Nie posiada on także edytora, co znacząco utrudnia tworzenie gier i~ogranicza jego zastosowania do tworzenia dem technologicznych i~testowania nowych technik usprawniających rendering. Ze względu na wspomniane problemy z~obsługą DirectX 11 został on odrzucony po~licznych próbach uruchomienia własnej aplikacji.\\

Zalety:
\begin{itemize}
\item bezpośredni dostęp do DirectX API (otwarty kod źródłowy silnika),
\item (teoretyczne) wsparcie dla DirectX 11.
\end{itemize}

Wady:
\begin{itemize}
\item brak edytora,
\item problemy ze wsparciem dla DirectX 11,
\item kolejne wersje dystrybuowane są wyjątkowo rzadko (ostatnia wersja sprzed dwóch lat).
\end{itemize}

\section{CryEngine 3 SDK}

\section{Unreal Engine 3}

\section{Unreal Development Kit (UDK)}

\section{Autodesk Maya}

\section{Autodesk 3D Studio Max}

\section{Blender}

\section{Gimp}

\section{Adobe Flash - Scaleform}

\section{Git}

W projektach nad którymi pracuje kilka osób niemożliwa jest praca bez systemu kontroli wersji. Początkowo, ze względu na pracę z kodem źródłowym, skorzystano z systemu Git ze względu na łatwość użycia oraz możliwość utworzenia prywatnego repozytorium na koncie studenckim w serwisie GitHub. Niestety ze względu na przejście na Unreal Development Kit konieczne było skorzystanie z narzędzia Perforce.\\

Zalety:
\begin{itemize}
\item łatwość obsługi,
\item dobra obsługa plików tekstowych,
\item prywatne repozytorium w serwisie GitHub dla studentów.
\end{itemize}

Wady:
\begin{itemize}
\item problemy z obsługą dużych plików binarnych,
\item brak integracji z Unreal Development Kit.
\end{itemize}

\section{Perforce}

Ostatnim opisywanym narzędziem jest system kontroli wersji Perforce, który dobrze obsługuje duże pliki binarne, dzięki czemu jest popularny w firmach zajmujących się wytwarzaniem oprogramowania (w tym gier). Pozwala on również -- w przeciwieństwie do gita -- na integrację z edytorem Unreal Development Kit, co znacząco przyspiesza i ułatwia pracę. Niestety ma on również swoje wady, które ujawniają się dopiero z upływem czasu, na przykład problemy z obsługą usuniętych plików (nadal uwzględniane są w listach zmian) czy nie zapisywanie zmian wprowadzonych w plikach na wspomniane listy.

Zalety:
\begin{itemize}
\item dobra obsługa plików binarnych,
\item integracja z Unreal Development Kit.
\end{itemize}

Wady:
\begin{itemize}
\item niespodziewane problemy wynikające z niewłaściwej obsługi zmian,
\item konieczność zainstalowania dedykowanego oprogramowania serwerowego i utrzymania serwera.
\end{itemize}

\section{Jira}

Przy pracy w kilkuosobowych zespołach przydaje się również oprogramowanie do zarządzania projektem. Skorzystano tu z platformy Jira. Umożliwia ona przydział zadań do poszczególnych członków zespołu. Ułatwia to pracę w wiele osób, poniewż zabezpiecza przed sytuacją, w~której kilka osób będzie wykonywało to samo zadanie. Wbudowany system szacowania czasu realizacji zadania pomaga oszacować czas potrzebny na stworzenie danej funkcjonalności i~zakończenie projektu. Jira posiada również możliwość zintegrowania z systemem kontroli wersji, co pozwala śledzić postęp prac, na podstawie komentarzy przy tworzeniu kolejnych rewizji.

Zalety:
\begin{itemize}
\item możliwość estymowania czasu wykonywania zadania
\item integracja z systemami kontroli wersji
\item duża liczba dodatków
\item definiowanie priorytetów zadań
\end{itemize}

Wady:
\begin{itemize}
\item tylko płatna licencja
\item nie do końca czytelny interfejs
\end{itemize}