
% Szkielet dla pracy inżynierskiej pisanej w języku polskim.

\documentclass[polish,bachelor,a4paper,oneside]{ppfcmthesis}

\usepackage{polski}
\usepackage[utf8]{inputenc}
\usepackage[OT4]{fontenc}


% Authors of the thesis here. Separate them with \and
\author{%
   Krzysztof~Marciniak \album{106574} \and 
   Piotr~Przybysz \album{106602} \and 
   Mikołaj~Szychowiak \album{106580} \and 
   Ryszard~Wojtkowiak \album{106609}}
\title{UJelly -- Game Design Document (GDD)}                   % Note how we protect the final title phrase from breaking
%\ppsupervisor{dr~inż.Witold Andrzejewski} % Your supervisor comes here.
\ppyear{2015}                                         % Year of final submission (not graduation!)


\begin{document}

% Front matter starts here
\frontmatter\pagestyle{empty}%
\maketitle\cleardoublepage%

% Blank info page for "karta dyplomowa"
\thispagestyle{empty}\vspace*{\fill}%
%\begin{center}Tutaj przychodzi karta pracy dyplomowej;\\oryginał wstawiamy do wersji dla archiwum PP, w pozostałych kopiach wstawiamy ksero.\end{center}%
\vfill\cleardoublepage%

% Table of contents.
\pagenumbering{Roman}\pagestyle{ppfcmthesis}%
\tableofcontents* \cleardoublepage%

% Main content of your thesis starts here.
\mainmatter%
\chapter{Wprowadzenie}

\section{Cel dokumentu}
Celem dokumentu jest przedstawienie wizji i~pełnego projektu gry pod roboczym
tytułem UJelly. Dokument ma za zadanie:
\begin{itemize}
\item zebrać wszystkie decyzje projektowe,
\item stanowić informacje dla zespołu realizującego grę,
\item służyć celom dydaktycznym (dla autorów i~innych).
\end{itemize}
Dokument stanowi również integralną część pracy inżynierskiej tworzonej przez autorów gry.

\section{Odbiorcy dokumentu}
Odbiorcami dokumentu są członkowie zespołu pracującego nad grą - osoby zaangażowane w~implementację i~testowanie, oraz osoby, które chciałyby poznać techniki projektowania gier.

\section{Przyjęte konwencje}
Tekst napisany kursywą dotyczy nazw własnych występujących w~otoczeniu „zwykłego” tekstu oraz nazwy tej gry.

\chapter{Informacje ogólne}

\section{Ogólny opis gry}
UJelly jest grą typu shooter z~widokiem z~perspektywy trzeciej osoby (Third-person shooter - TPS), w~której gracz wciela się w~postać, będącą żelkiem, wyposażonym w~broń, którego celem jest zabicie jak największą liczbę razy innych graczy. Gra nie posiada fabuły, ani konwencji. Ponieważ plansze, na których toczy się gra mogą być tworzone przez graczy, więc sami gracze mogą tworzyć konwencje lub elementy fabularne.

Gracz może atakować innych graczy poprzez frontalny atak, zaczajenie się, ma również możliwość ucieczki.

\section{Motywacja}
Głównym powodem powstania gry jest chęć zdobycia doświadczenia w~tworzeniu gier. W~celu zwiększenia szansy ukończenia projektu było konieczne podjęcie pewnych decyzji projektowych. Innym celem tworzenia projektu jest poznanie opinii na temat stworzonej gry. Istotnym elementem jest również wyciągnięcie wniosków z~procesu tworzenia projektu.

W~trakcie tworzenia gry nie wzorowaliśmy się na żadnym konkretnym tytule, ale można dostrzec podobieństwa do takich gier jak: \textit{Quake}, \textit{Counter-Strike}, \textit{Unreal Tournament III} czy też oryginalne podejście do bohaterów gry tak jak w~serii gry \textit{Worms}.

\section{Platformy docelowe}
Platformą docelową są komputery osobiste (PC) z~systemem operacyjnym Microsoft Windows XP z~dodatkiem Service Pack 2 lub nowszy, wyposażone w~procesor o~częstotliwości taktowania 2.4 GHz, 1 GB pamięci RAM oraz kartę graficzną zgodną z~Microsoft DirectX~11.

\section{Odbiorcy docelowi}
Docelową grupą odbiorców są osoby lubiące gry typu shooter, szukające czegoś innego niż najpopularniejsze tego typu np. \textit{Call of Duty} czy \textit{CounterStrike}, należące do grupy graczy okazjonalnych. Odbiorca gry posiada poczucie humoru. Z~uwagi na charakter gry, zawartą w~niej minimalną przemoc, przeznaczona jest dla osób powyżej 13 roku życia\footnote{Zgodnie z~warunkami ocen ESRB}.

\section{Decyzje technologiczne}
W~tracie projektowania podjęto następujące decyzje dotyczące środowiska, w~którym ma działać gra, oraz rozgrywki:
\begin{itemize}
\item Microsoft Windows -- popularny system operacyjny na komputery osobiste, jest kompatybilny z~wieloma innymi użytymi narzędziami oraz wymagany do zgłoszenia gry do Imagine Cup.
\item UScript -- język skryptowy przeznaczony do tworzenia skryptów w~Unreal Development Kit. Przewiduje się, że język ten, wraz z~wykorzystywanym środowiskiem, jest wystarczający do stworzenia opisywanej gry.
\item Wieloosobowa gra przez sieć - gra sieciowa przeciwko innemu graczowi jest bardziej emocjonująca, dodatkowo żadna sztuczna inteligencja nie dorówna przeciętnemu graczowi w~tworzeniu nowych, niepowtarzalnych stanów gry. Gra z~innymi ludźmi jest często nieprzewidywalna, a~więc ciekawsza.
\item Unreal Development Kit -- popularne narzędzie do tworzenia gier, dostępne bez opłat dla celów niekomercyjnych. Jest rozbudowane oraz spełnia wszystkie wymagania stawiane przez twórców silnikowi gry, takie jak: mała awaryjność, łatwość modyfikowania, tworzenia nowych plansz.
\item DirectX 11 -- wykorzystanie tej technologii pozwala zarówno na podniesienie ogólnej jakości grafiki (efekt bokeh depth of field) jak i zwiększenie wydajności przy dużej liczbie źródeł światła przez wykorzystanie techniki deferred shading,
\item Kamera TPP -- użycie kamery TPP pozwala na lepsze wyeksponowanie efektów graficznych uzyskanych przy pomocy DirectX 11.
\end{itemize}

Problemy i~wymagania, jakie wiążą się z~podjętymi decyzjami:
\begin{itemize}
\item Sterowanie musi być proste i~intuicyjne, wykorzystujące klawiaturę i~myszkę, czyli najpopularniejsze urządzenia wejścia komputera
\item Należy mieć na uwadze obciążenie serwerów. Ze względu na charakterystykę gry należy monitorować liczbę użytkowników na danej planszy.
\item Konieczne jest stworzenie estetycznych, trójwymiarowych modeli postaci oraz realistycznych animacji, aby gra zachęcała odbiorców warstwą wizualną.
\end{itemize}

\section{Wykorzystywane licencje}
Do realizacji gry nie będzie konieczne nabycie żadnych licencji związanych z~podjętymi decyzjami technologicznymi. Unreal Development Kit wykorzystywany jest w~celach niekomercyjnych, więc nie wymaga zakupu licencji. Pozostałe oprogramowanie potrzebne do stworzenia gry jest dostępne bez opłat na licencji edukacyjnej.


\chapter{Interfejs użytkownika}

\section{Obsługa gry}
UJelly jest grą przeznaczoną na komputery osobiste. Stosujemy już sprawdzone rozwiązania, jak poruszanie się w~przód, tył i~na boki przy pomocy przycisków ,,WSAD", przycisk ,,spacja" do skoków oraz strzelanie przy pomocy prawego i~lewego przycisku myszy. 
Podsumowanie obecnej rozgrywki można podejrzeć naciskając przycisk F1, w~trakcie rozgrywki można również wybrać broń przy pomocy przycisków numerycznych lub poprzez kręcenie rolką myszy. Poprzez naciśnięcie przycisku “ESC” można przejść do menu obsługiwanego poprzez kursor myszy.

\section{HUD}
Użytkownik aplikacji widzi obok modelu postaci wykresy przedstawiające liczbę punktów życia, a także rodzaj broni używanej. Jest to standardowy jak dla gier z gatunku strzelanek model. Interfejs użytkownika nie zmienia położenia w trakcie rozgrywki.

\section{Praca kamery}
Gracz widzi swoją postać z~perspektywy trzeciej osoby (TPP). Cała plansza, po której porusza się postać gracza jest w~pełni trójwymiarowa. Kamera podąża za celownikiem umieszczonym centralnie na ekranie. W~przypadku, gdy za graczem jest obiekt, to kamera powinna się przesuwać tak, aby widok był z~poza obiektu, który zawadza.


\chapter{Mechanika podstawowa}

\section{Personifikacja gracza}
Gracz steruje postacią będącą jednym z~wielu żelków na planszy. Gracz może zabijać innych graczy poprzez strzelanie do nich. Bohater może zginąć w wyniku ataku innych graczy, upadku z~dużej wysokości, wypadnięciu poza planszę lub gdy gracz użyje broni JellyCanon zbyt dużą liczbę razy. 

\section{Poruszanie się}
Gra będzie toczyć się na planszy, po której Gracz może przemieszczać się poprzez chodzenie po obiektach, wskakiwanie i~wspinanie się na nie. 
W~przyszłości, w~kolejnych rozszerzeniach gry, może być możliwe bezpośrednie oddziaływanie gracza na otoczenie takie jak: niszczenie obiektów, budowanie obiektów, mostów, drabin.

\section{Rozgrywka}
Rozgrywka polega na odnajdowaniu i~zabijaniu innych graczy. Za każde zabójstwo gracz dostaje punkt, każda śmierć gracza jest również odnotowywana.

\section{Śmierć gracza}
Gracz po śmierci musi odczekać pewną ilość czasu, by móc wrócić do rozgrywki. Po śmierci gracz upuszcza broń, która po podniesieniu pozwala na przywrócenie części punktów zdrowia podnoszącemu. W~wyniku takiego podnoszenia, będzie możliwość przekroczenia startowej liczby punktów zdrowia.

\section{Punkt startowy}
Na każdym poziomie znajdują się punkty startowe, służące do rozpoczęcia gry po śmierci Gracza lub na początku rozgrywki. Punkt ten powinien być w~miejscu bezpiecznym, aby gracz rozpoczynający rozgrywkę lub wracający do gry po śmierci nie zginął ponownie z~powodu upadku, czy też wypadnięcia poza planszę. Dwa punkty startowe nie powinny również znajdować się zbyt blisko siebie.

\section{Dostępne bronie}
Gracz będzie miał do wyboru dwie bronie. Jedną z~nich jest JellyCannon, która służy do ataku na odległość, ale jej wadą jest to, że każdy pocisk wymaga poświęcenia punktów zdrowia. Drugą z~dostępnych broni będzie JellySpearOrPunch, która nie wymaga poświęcenia punktów zdrowia, ale jej wadą jest to, że należy się zbliżyć na niewielką odległość do obiektu atakowanego.

W~przyszłości, w~kolejnych rozszerzeniach gry, może być możliwe dodanie innych rodzajów broni.

\section{Warunki zakończenia gry}
Aby zakończyć daną rozgrywkę jeden z~graczy musi zebrać określoną liczbę punktów. Na końcu rozgrywki wyświetlany jest ranking, gdzie gracze są uszeregowani wg. liczby zgromadzonych punktów.


\chapter{Informacje dodatkowe}

\section{Czynniki atrakcyjne dla graczy}
Gra UJelly w~założeniach jest grą pozwalającą autorom rozwinąć się w~dziedzinie tworzenia gier oraz poznać technologię z~tym związane. Projekt nie jest oryginalny, realizuje prosty i~znany schemat rozgrywki jeden przeciwko wszystkim (Deathmatch), znany z~takich tytułów jak \textit{CounterStrike} czy \textit{Quake}. Reguły tego typu rozgrywki są przejrzyste, pozwalają na tworzenie rankingów, porównywanie wyników z~innymi graczami.

Projekt gry zawiera wiele elementów, które mogą sprawić, że gra zostanie dobrze przyjęta przez odbiorców. Rankingi graczy zachęcają do tego, by być w każdej rozgrywce jak najlepszym, a~żeby osiągnąć sukces i~zwyciężyć w~rozgrywce należy lepiej poznać planszę i~więcej grać z~innymi graczami, aby poznawać ich strategie.
W~grze istnieje możliwość dodawania nowych plansz, oraz broni, które mogą zmieniać strategię stosowaną w~poszczególnych rozgrywkach.
W~aktualnej implementacji gry, gracz powinien przewidywać czy lepiej zabić gracza z~dystansu, czy jednak pokonanie odległości dzielącej od przeciwnika i~zabicie go bronią krótko dystansową nie będzie skutkowało utratą mniejszej ilości punktów zdrowia, co skutkuje większą szansą na wyższe miejsce w rankingu.

Gra jest planowana jako darmowa, co jest również pozytywnym czynnikiem dla gracza.

\section{Zagrożenia projektowe}
Wielu czynników, które przyczyniają się do zwiększenia ryzyka i~zagrożeń nie ma znaczenia, jeżeli produkcja jest darmowa. Warto jednak nie ignorować możliwości zainteresowania graczy tytułem, aby móc lepiej realizować projekty gier w przyszłości.

Kluczowym czynnikiem decydującym o~popularności gry, jest projekt plansz oraz broni na nich dostępnych. Plansze muszą być grywalne, muszą znajdować się tam więcej niż jedna możliwa ścieżka dojścia do przeciwnika, muszą być elementy gdzie można się schować lub zaczaić się na innych graczy. Sama broń dostępna w~danej planszy musi wspierać i~eksponować strategie dostępne na planszy, a~także dzięki unikalnym właściwościom może tworzyć nowe.

W kolejnych rozszerzeniach gry planowane jest dodanie miejsc specjalnych na planszach, które będą zadawać, lub leczyć obrażenia, przemieszczać gracza, lub dawać mu tymczasową przewagę w~grze taką jak niewidzialność czy dodatkowe obrażenia.

Bardzo ważnym czynnikiem, na który twórcy mają tylko pośredni wpływ jest społeczność graczy, bo to od niej zależy jak dużo będzie serwerów, rozgrywek i~turniejów. Na początku warto stworzyć forum gry, by dać graczom miejsce do rozmów i~komunikacji, oraz dać narzędzia twórcom do wpływu na kształt społeczności, która będzie się tworzyć wokół gry.

Kłopotliwe może się okazać przygotowanie modeli i~tekstur, które będą estetyczne i~całościowo współgrały ze sobą. Jest to problematyczne ze względu na brak posiadania wyspecjalizowanej osoby, która w~momencie rozpoczęcia pracy nad grą posiadałaby już wystarczające umiejętności. Planujemy jednak, by jeden z twórców w trakcie rozwijania gry rozwinął swoje umiejętności wystarczająco, aby stworzyć pierwszą wersję gry.

Mimo braku opłat za grę, może być trudno wypromować grę na tyle, by społeczność wokół tej gry zaistniała i~zaczęła się rozwijać bez ingerencji twórców. Na początku promocji gry będziemy się starać dotrzeć do jak największej liczby odbiorców w Polsce i~na świecie, przy pomocy serwisów, rekomendacji czy poprzez promowanie gry poprzez konkursy np.~Imagine~Cup. 

\section{Ogólny plan promocji gry (Imagine Cup)}
Promocje gry rozpoczyna się od udziału w~konkursie \textit{Imagine Cup}, gdzie w~szerszym gronie można zaprezentować rozgrywkę. Do tego czasu będzie skończone forum, na którym będzie rozwijać się początkowo społeczność wokół tej gry. Będzie to wymagało na początku wiele wkładu pracy moderatorskiej, ale od pewnego momentu powinni pojawić się moderatorzy ze społeczności, odciążając tym samym twórców. W~celach promocji użyta zostaną platformy społecznościowe \textit{Facebook}, \textit{Google+}, by trafić do jak najszerszego grona odbiorców i~przekazać im materiały promocyjne, prezentacje rozgrywki, informacje o~organizowanych akcjach. Planowane są również recenzje gry w~serwisach takich jak \url{http://polter.pl/}, \url{http://www.gry-online.pl/}. Jedną z~zalet gry są charakterystyczni bohaterowie, istnieje zatem możliwość promocji samej gry przy pomocy np. rozdawania żelkowych misiów.


% All appendices and extra material, if you have any.
\cleardoublepage\appendix%
\chapter{Referencje}
\begin{enumerate}
\item Informacje o serii gier \textit{Worms}, \url{http://pl.wikipedia.org/wiki/Worms_%28seria%29}
\item Informacje o grze \textit{Quake}, \url{http://pl.wikipedia.org/wiki/Quake}
\item Informacje o grze \textit{Counter-Strike}, \url{http://pl.wikipedia.org/wiki/Counter-Strike}
\item Informacje o grze \textit{Unreal Tournament III}, \url{http://pl.wikipedia.org/wiki/Unreal_Tournament_III}
\item Adams Ernest, \textit{Projektowanie Gier. Podstawy. Wydanie II}, HELION, 2011, ISBN: 978-83-246-2781-3
\end{enumerate}



% Bibliography (books, articles) starts here.
\bibliographystyle{plalpha}{\raggedright\sloppy\small\bibliography{bibliography}}

% Colophon is a place where you should let others know about copyrights etc.
\ppcolophon

\end{document}
