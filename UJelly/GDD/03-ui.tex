\chapter{Interfejs użytkownika}

\section{Obsługa gry}
UJelly jest grą przeznaczoną na komputery osobiste. Stosujemy już sprawdzone rozwiązania, jak poruszanie się w~przód, tył i~na boki przy pomocy przycisków ,,WSAD", przycisk ,,spacja" do skoków oraz strzelanie przy pomocy prawego i~lewego przycisku myszy. 
Podsumowanie obecnej rozgrywki można podejrzeć naciskając przycisk F1, w~trakcie rozgrywki można również wybrać broń przy pomocy przycisków numerycznych lub poprzez kręcenie rolką myszy. Poprzez naciśnięcie przycisku “ESC” można przejść do menu obsługiwanego poprzez kursor myszy.

\section{HUD}
Użytkownik aplikacji widzi obok modelu postaci wykresy przedstawiające liczbę punktów życia, a także rodzaj broni używanej. Jest to standardowy jak dla gier z gatunku strzelanek model. Interfejs użytkownika nie zmienia położenia w trakcie rozgrywki.

\section{Praca kamery}
Gracz widzi swoją postać z~perspektywy trzeciej osoby (TPP). Cała plansza, po której porusza się postać gracza jest w~pełni trójwymiarowa. Kamera podąża za celownikiem umieszczonym centralnie na ekranie. W~przypadku, gdy za graczem jest obiekt, to kamera powinna się przesuwać tak, aby widok był z~poza obiektu, który zawadza.

