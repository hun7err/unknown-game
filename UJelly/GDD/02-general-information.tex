\chapter{Informacje ogólne}

\section{Ogólny opis gry}
UJelly jest grą typu shooter z~widokiem z~perspektywy trzeciej osoby (Third-person shooter - TPS), w~której gracz wciela się w~postać, będącą żelkiem, wyposażonym w~broń, którego celem jest zabicie jak największą liczbę razy innych graczy. Gra nie posiada fabuły, ani konwencji. Ponieważ plansze, na których toczy się gra mogą być tworzone przez graczy, więc sami gracze mogą tworzyć konwencje lub elementy fabularne.

Gracz może atakować innych graczy poprzez frontalny atak, zaczajenie się, ma również możliwość ucieczki.

\section{Motywacja}
Głównym powodem powstania gry jest chęć zdobycia doświadczenia w~tworzeniu gier. W~celu zwiększenia szansy ukończenia projektu było konieczne podjęcie pewnych decyzji projektowych. Innym celem tworzenia projektu jest poznanie opinii na temat stworzonej gry. Istotnym elementem jest również wyciągnięcie wniosków z~procesu tworzenia projektu.

W~trakcie tworzenia gry nie wzorowaliśmy się na żadnym konkretnym tytule, ale można dostrzec podobieństwa do takich gier jak: \textit{Quake}, \textit{Counter-Strike}, \textit{Unreal Tournament III} czy też oryginalne podejście do bohaterów gry tak jak w~serii gry \textit{Worms}.

\section{Platformy docelowe}
Platformą docelową są komputery osobiste (PC) z~systemem operacyjnym Microsoft Windows XP z~dodatkiem Service Pack 2 lub nowszy, wyposażone w~procesor o~częstotliwości taktowania 2.4 GHz, 1 GB pamięci RAM oraz kartę graficzną zgodną z~Microsoft DirectX~11.

\section{Odbiorcy docelowi}
Docelową grupą odbiorców są osoby lubiące gry typu shooter, szukające czegoś innego niż najpopularniejsze tego typu np. \textit{Call of Duty} czy \textit{CounterStrike}, należące do grupy graczy okazjonalnych. Odbiorca gry posiada poczucie humoru. Z~uwagi na charakter gry, zawartą w~niej minimalną przemoc, przeznaczona jest dla osób powyżej 13 roku życia\footnote{Zgodnie z~warunkami ocen ESRB}.

\section{Decyzje technologiczne}
W~tracie projektowania podjęto następujące decyzje dotyczące środowiska, w~którym ma działać gra, oraz rozgrywki:
\begin{itemize}
\item Microsoft Windows -- popularny system operacyjny na komputery osobiste, jest kompatybilny z~wieloma innymi użytymi narzędziami oraz wymagany do zgłoszenia gry do Imagine Cup.
\item UScript -- język skryptowy przeznaczony do tworzenia skryptów w~Unreal Development Kit. Przewiduje się, że język ten, wraz z~wykorzystywanym środowiskiem, jest wystarczający do stworzenia opisywanej gry.
\item Wieloosobowa gra przez sieć - gra sieciowa przeciwko innemu graczowi jest bardziej emocjonująca, dodatkowo żadna sztuczna inteligencja nie dorówna przeciętnemu graczowi w~tworzeniu nowych, niepowtarzalnych stanów gry. Gra z~innymi ludźmi jest często nieprzewidywalna, a~więc ciekawsza.
\item Unreal Development Kit -- popularne narzędzie do tworzenia gier, dostępne bez opłat dla celów niekomercyjnych. Jest rozbudowane oraz spełnia wszystkie wymagania stawiane przez twórców silnikowi gry, takie jak: mała awaryjność, łatwość modyfikowania, tworzenia nowych plansz.
\item DirectX 11 --
\item Kamera TPP -- użycie kamery TPP pozwala na lepsze wyeksponowanie efektów graficznych uzyskanych przy pomocy DirectX 11.
\end{itemize}

Problemy i~wymagania, jakie wiążą się z~podjętymi decyzjami:
\begin{itemize}
\item Sterowanie musi być proste i~intuicyjne, wykorzystujące klawiaturę i~myszkę, czyli najpopularniejsze urządzenia wejścia komputera
\item Należy mieć na uwadze obciążenie serwerów. Ze względu na charakterystykę gry należy monitorować liczbę użytkowników na danej planszy.
\item Konieczne jest stworzenie estetycznych, trójwymiarowych modeli postaci oraz realistycznych animacji, aby gra zachęcała odbiorców warstwą wizualną.
\end{itemize}

\section{Wykorzystywane licencje}
Do realizacji gry nie będzie konieczne nabycie żadnych licencji związanych z~podjętymi decyzjami technologicznymi. Unreal Development Kit wykorzystywany jest w~celach niekomercyjnych, więc nie wymaga zakupu licencji. Pozostałe oprogramowanie potrzebne do stworzenia gry jest dostępne bez opłat na licencji edukacyjnej.

