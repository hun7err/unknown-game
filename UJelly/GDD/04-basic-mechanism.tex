\chapter{Mechanika podstawowa}

\section{Personifikacja gracza}
Gracz steruje postacią będącą jednym z~wielu żelków na planszy. Gracz może zabijać innych graczy poprzez strzelanie do nich. Bohater może zginąć w wyniku ataku innych graczy, upadku z~dużej wysokości, wypadnięciu poza planszę lub gdy gracz użyje broni JellyCanon zbyt dużą liczbę razy. 

\section{Poruszanie się}
Gra będzie toczyć się na planszy, po której Gracz może przemieszczać się poprzez chodzenie po obiektach, wskakiwanie i~wspinanie się na nie. 
W~przyszłości, w~kolejnych rozszerzeniach gry, może być możliwe bezpośrednie oddziaływanie gracza na otoczenie takie jak: niszczenie obiektów, budowanie obiektów, mostów, drabin.

\section{Rozgrywka}
Rozgrywka polega na odnajdowaniu i~zabijaniu innych graczy. Za każde zabójstwo gracz dostaje punkt, każda śmierć gracza jest również odnotowywana.

\section{Śmierć gracza}
Gracz po śmierci musi odczekać pewną ilość czasu, by móc wrócić do rozgrywki. Po śmierci gracz upuszcza broń, która po podniesieniu pozwala na przywrócenie części punktów zdrowia podnoszącemu. W~wyniku takiego podnoszenia, będzie możliwość przekroczenia startowej liczby punktów zdrowia.

\section{Punkt startowy}
Na każdym poziomie znajdują się punkty startowe, służące do rozpoczęcia gry po śmierci Gracza lub na początku rozgrywki. Punkt ten powinien być w~miejscu bezpiecznym, aby gracz rozpoczynający rozgrywkę lub wracający do gry po śmierci nie zginął ponownie z~powodu upadku, czy też wypadnięcia poza planszę. Dwa punkty startowe nie powinny również znajdować się zbyt blisko siebie.

\section{Dostępne bronie}
Gracz będzie miał do wyboru dwie bronie. Jedną z~nich jest JellyCannon, która służy do ataku na odległość, ale jej wadą jest to, że każdy pocisk wymaga poświęcenia punktów zdrowia. Drugą z~dostępnych broni będzie JellySpearOrPunch, która nie wymaga poświęcenia punktów zdrowia, ale jej wadą jest to, że należy się zbliżyć na niewielką odległość do obiektu atakowanego.

W~przyszłości, w~kolejnych rozszerzeniach gry, może być możliwe dodanie innych rodzajów broni.

\section{Warunki zakończenia gry}
Aby zakończyć daną rozgrywkę jeden z~graczy musi zebrać określoną liczbę punktów. Na końcu rozgrywki wyświetlany jest ranking, gdzie gracze są uszeregowani wg. liczby zgromadzonych punktów.

