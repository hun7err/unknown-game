\chapter{Informacje dodatkowe}

\section{Czynniki atrakcyjne dla graczy}
Gra UJelly w~założeniach jest grą pozwalającą autorom rozwinąć się w~dziedzinie tworzenia gier oraz poznać technologię z~tym związane. Projekt nie jest oryginalny, realizuje prosty i~znany schemat rozgrywki jeden przeciwko wszystkim (Deathmatch), znany z~takich tytułów jak \textit{CounterStrike} czy \textit{Quake}. Reguły tego typu rozgrywki są przejrzyste, pozwalają na tworzenie rankingów, porównywanie wyników z~innymi graczami.

Projekt gry zawiera wiele elementów, które mogą sprawić, że gra zostanie dobrze przyjęta przez odbiorców. Rankingi graczy zachęcają do tego, by być w każdej rozgrywce jak najlepszym, a~żeby osiągnąć sukces i~zwyciężyć w~rozgrywce należy lepiej poznać planszę i~więcej grać z~innymi graczami, aby poznawać ich strategie.
W~grze istnieje możliwość dodawania nowych plansz, oraz broni, które mogą zmieniać strategię stosowaną w~poszczególnych rozgrywkach.
W~aktualnej implementacji gry, gracz powinien przewidywać czy lepiej zabić gracza z~dystansu, czy jednak pokonanie odległości dzielącej od przeciwnika i~zabicie go bronią krótko dystansową nie będzie skutkowało utratą mniejszej ilości punktów zdrowia, co skutkuje większą szansą na wyższe miejsce w rankingu.

Gra jest planowana jako darmowa, co jest również pozytywnym czynnikiem dla gracza.

\section{Zagrożenia projektowe}
Wielu czynników, które przyczyniają się do zwiększenia ryzyka i~zagrożeń nie ma znaczenia, jeżeli produkcja jest darmowa. Warto jednak nie ignorować możliwości zainteresowania graczy tytułem, aby móc lepiej realizować projekty gier w przyszłości.

Kluczowym czynnikiem decydującym o~popularności gry, jest projekt plansz oraz broni na nich dostępnych. Plansze muszą być grywalne, muszą znajdować się tam więcej niż jedna możliwa ścieżka dojścia do przeciwnika, muszą być elementy gdzie można się schować lub zaczaić się na innych graczy. Sama broń dostępna w~danej planszy musi wspierać i~eksponować strategie dostępne na planszy, a~także dzięki unikalnym właściwościom może tworzyć nowe.

W kolejnych rozszerzeniach gry planowane jest dodanie miejsc specjalnych na planszach, które będą zadawać, lub leczyć obrażenia, przemieszczać gracza, lub dawać mu tymczasową przewagę w~grze taką jak niewidzialność czy dodatkowe obrażenia.

Bardzo ważnym czynnikiem, na który twórcy mają tylko pośredni wpływ jest społeczność graczy, bo to od niej zależy jak dużo będzie serwerów, rozgrywek i~turniejów. Na początku warto stworzyć forum gry, by dać graczom miejsce do rozmów i~komunikacji, oraz dać narzędzia twórcom do wpływu na kształt społeczności, która będzie się tworzyć wokół gry.

Kłopotliwe może się okazać przygotowanie modeli i~tekstur, które będą estetyczne i~całościowo współgrały ze sobą. Jest to problematyczne ze względu na brak posiadania wyspecjalizowanej osoby, która w~momencie rozpoczęcia pracy nad grą posiadałaby już wystarczające umiejętności. Planujemy jednak, by jeden z twórców w trakcie rozwijania gry rozwinął swoje umiejętności wystarczająco, aby stworzyć pierwszą wersję gry.

Mimo braku opłat za grę, może być trudno wypromować grę na tyle, by społeczność wokół tej gry zaistniała i~zaczęła się rozwijać bez ingerencji twórców. Na początku promocji gry będziemy się starać dotrzeć do jak największej liczby odbiorców w Polsce i~na świecie, przy pomocy serwisów, rekomendacji czy poprzez promowanie gry poprzez konkursy np.~Imagine~Cup. 

\section{Ogólny plan promocji gry (Imagine Cup)}
Promocje gry rozpoczyna się od udziału w~konkursie \textit{Imagine Cup}, gdzie w~szerszym gronie można zaprezentować rozgrywkę. Do tego czasu będzie skończone forum, na którym będzie rozwijać się początkowo społeczność wokół tej gry. Będzie to wymagało na początku wiele wkładu pracy moderatorskiej, ale od pewnego momentu powinni pojawić się moderatorzy ze społeczności, odciążając tym samym twórców. W~celach promocji użyta zostaną platformy społecznościowe \textit{Facebook}, \textit{Google+}, by trafić do jak najszerszego grona odbiorców i~przekazać im materiały promocyjne, prezentacje rozgrywki, informacje o~organizowanych akcjach. Planowane są również recenzje gry w~serwisach takich jak \url{http://polter.pl/}, \url{http://www.gry-online.pl/}. Jedną z~zalet gry są charakterystyczni bohaterowie, istnieje zatem możliwość promocji samej gry przy pomocy np. rozdawania żelkowych misiów.
